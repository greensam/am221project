\documentclass[12pt]{article}
\usepackage{fullpage}
\usepackage[urlcolor=blue]{hyperref}
\usepackage{fullpage,hyperref }

\begin{document}
\begin{center}
{\Large Applied Mathmamatics 221}\\
Project Milestone 1: Proposal\\
David Freed and Sam Green\\
\today
\end{center}

\section{Problem Statement}

In this project, we intend to model the predictive properties of Twitter data relating to outcomes of high visibility basketball games in the NBA or NCAA. Previous work has applied tools from statistical analysis to establish that Twitter data can be used to predict outcomes of games in sporting events\footnote{Previous work on the NFL: \url{https://www.cs.cmu.edu/~nasmith/papers/sinha+dyer+gimpel+smith.mlsa13.pdf}}. Previous work that we have discovered, however, leaves unanswered questions about how the predictive quality of Twitter data changes over time. In this project, we will investigate the shape of the ``predictive curve'' for tweets, with the goal of answering the question: ``At what point in a game do tweets become an accurate predictor of outcomes?''

The project has a fundamentally applied goal but nonetheless depends substantially
on optimization theory. We intend to build a predictive model that investigates
outcomes of games, but the main question of interest is rooted in sensitivity
analysis. Rather than simply asking ``Can we use Twitter data to predict outcomes?'',
we wish to perform sensitivity analysis that characterizes when the model 
we produce becomes useful. 

The pool of questions we can pose is also very extensible. After building 
a baseline predictive model, we can alter the character of the input data
(limiting to tweets from specific types of Twitter users, correlating predictive
quality of tweets with geolocation, testing the number of tweets necessary 
for the model to become predictive.) 

\section{Deliverables}

\subsection{Timeline: Mid-March}  
\begin{itemize}
\item Twitter data collection script written (using the Twitter Streaming API)
\item Game data sourced, and collection script written (game data must be converted to time series)
\end{itemize} 

\subsection{Timeline: End of March}
\begin{itemize}
\item Data collected (Twitter and Game)
\item Sentiment classification script for Tweets
\item Baseline expert prediction set created (for benchmarking)
\end{itemize}

\subsection{Mid-April} 
\begin{itemize}
\item Statistical Model in process or complete.
\item Preliminary sensitivity analysis \& results.
\end{itemize}

\subsection{End of April} 
\begin{itemize}
\item Robustness checks.
\item Final Analysis.
\item Extensions.
\end{itemize}

\section{Collaboration}

This project will be completed by David Freed and Sam Green. We intend to work collaboratively on all aspects of the project and will divide specific implementation tasks based on our previous experience as they arise.


\section{Data}

We will primarily collect two types of data for this project: Twitter-based data and outcome-based data. As illustrated in the above problem statement, our goal is to create connections between the two and see how good Twitter is as an inferential model (i.e. given past game outcomes, how good is Twitter at predicting future outcomes). 

\subsection{Twitter Data}

Our primary aim is to collect data from Twitter regarding an actual game. We will sort our data by time, date, and hashtag. Since we are interested in overall sentiment during the game, we will limit our search to tweets between tipoff and the final buzzer, data that will come from the following dataset. 

We will further filter by hashtag/account to get relevant Tweets. In a game between the Cleveland Cavaliers and Golden State Warriors, for example, we might look at all Tweets containing \#GSW, \#Cavaliers, \#Dubs, \#GSWvCLE, and any number of relevant hashtags. It is not clear how we will select appropriate hashtags?with luck, we can create a script that will automate this process effectively. 

For each tweet, we want to grab the following information:
\begin{enumerate}
	\item User characteristics (tagline, number of followers, etc.)
	\item Tweet characteristics (time, location, no. of retweets/favorites, etc.)
	\item Sensitivity score
\end{enumerate}

\noindent
The last item will be the most salient characteristic of the Tweet--we want to know how the user is feeling about the game. We will get this sensitivity score by developing our own natural language processor. In order to do this effectively, we hope to borrow from the literature here.

\subsection{Box Score Data}

To connect this Twitter data with outcomes, we need to get a set of outcomes. We will create a simple scraper for box scores so that we can get the outcomes for all the games that we study. We envision scraping sites like \url{http://espn.go.com/nba/playbyplay?gameId=400828584} to get the following data for every game:
\begin{enumerate}
	\item Time of every score change (so as to keep track of score over time)
	\item Information about each play (who made which play)
\end{enumerate}

\noindent
Eventually, we will try to map the time that the event occurred in the game to when it appeared in real life. This will allow us to get a sense for what information is included in Tweets and what types of plays prompt people to get excited on Twitter.

\section{Next Steps}

Our next steps all have to do with acquiring data. We break it down here into steps needed to acquire Twitter data and steps needed to acquire box score data:

\subsubsection*{Twitter Data Next Steps}
\begin{enumerate}
	\item Get Twitter API keys
	\item Identify the best ways to get requests and the structure for streaming data from Twitter
	\item Read the literature on sensitivity analyses on Tweets
	\item Get trial data (ideally with location tags attached)
\end{enumerate}

\subsubsection*{Box Score Data Next Steps}
\begin{enumerate}
	\item Create a scrape to get all of the box score data
	\item Write a win probability model
\end{enumerate}

\noindent
The win probability model will likely be a replication of what people already do online and will provide a reference point for how good Twitter is at predicting games. We might initialize the model to the Vegas line or some other ex-ante predictor of performance (Pythagorean estimates are a commonly accepted way of measuring team strength). 

\end{document}