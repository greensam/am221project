\documentclass[12pt]{article}
\usepackage[hmargin=1in,vmargin=1in]{geometry}
\usepackage{amsmath}
\usepackage{setspace}
\usepackage{graphicx}
\usepackage{sectsty}
\usepackage{fancyvrb}
\usepackage{mathtools}
\usepackage{float}
\usepackage[flushleft]{threeparttable}

% Make all of the environments for different proof elements
\newtheorem{theorem}{Theorem}[section]
\newtheorem{lemma}[theorem]{Lemma}
\newtheorem{proposition}[theorem]{Proposition}
\newtheorem{corollary}[theorem]{Corollary}
\newenvironment{proof}[1][Proof:]{\begin{trivlist}
\item[\hskip \labelsep {\bfseries #1}]}{\end{trivlist}}
\newenvironment{definition}[1][Definition]{\begin{trivlist}
\item[\hskip \labelsep {\bfseries #1}]}{\end{trivlist}}
\newenvironment{example}[1][Example]{\begin{trivlist}
\item[\hskip \labelsep {\bfseries #1}]}{\end{trivlist}}
\usepackage{hyperref}
\usepackage[usenames, dvipsnames]{color}

\definecolor{cuse}{RGB}{212, 69, 0}
\definecolor{uva}{RGB}{0, 55, 119}

% Change back to use Arabic footnotes
\renewcommand*{\thefootnote}{\arabic{footnote}}
\sectionfont{\large}
\subsectionfont{\normalsize}

% Start the document
\begin{document}
\setcounter{page}{1}
\begin{doublespacing}

\title{Draft Literature Review}
\author{David Freed and Samuel Green}
\date{May 1, 2016}
\maketitle

% Literature Review
\section{Overview of Related Literature}

Previous literature has established that useful modeling information can be
derived from Twitter data. We begin by reviewing some literature
on sentiment classification for Twitter and then reference
previous work on using Twitter data for sports analysis. 

Previous work has established that accurate classification
is possible for Twitter data. Multiple different methods
have been shown to be reliable in practice, with multiple
different approaches used to train models. Go et al. provide
a useful overview and empirical work on
classification of Twitter data. Their work demonstrated
that reasonably high classification rates can be achieved 
using Naive Bayes, Maximum Entropy, and Support Vector Machin
approaches. Their work also showed that a reasonablely
accurate training set could be derived without hand-labeling 
using information embedded in Tweets, though we do not reapply
that result here (Go et al., 2009)\footnote{https://cs.stanford.edu/people/alecmgo/papers/TwitterDistantSupervision09.pdf}. 
Ibrahim and Yusoff also provided evidence recently that
reasonably accurate Naive Bayes classifiers could be 
trained using unexpectedly small datasets of labeled tweets, 
lending weight to the hypothesis that a large training
set should provide even more useful information, as in
the setting of this project (Ibrahim and Yusoff, 2015).\footnote{http://ieeexplore.ieee.org.ezp-prod1.hul.harvard.edu/stamp/stamp.jsp?tp=\&arnumber=7403510}. 


Twitter sentiment has also been applied for sports analytics, 
though as far as we can tell, not in a real-time predictive 
model. Sinha et. al. assembled a large dataset of
tweets related to NFL games in advance of these games
taking place. They used the datasets to inform
ex ante predictions about the outcomes of the games, 
after diong some classification work. They found
that Twitter could be used effectively to 
build a potentially profitable winner with the spread
prediction model. Their model produced a success rate of higher than 55\%, the 
benchmark needed to turn a profit after factoring in
bookmaker commissions. Sinha et. al. also built 
their dataset using targeted collections of hashtags, 
an approach that we re-apply to build our dataset 
(Sinha et al, 2013)\footnote{http://arxiv.org/abs/1310.6998}. 


Finally, the usefulness of Twitter in real-time
identification of events during sports games
has been established by previous work
A real-time system was built in 2012 by Zhao et. al., 
which was able to use Twitter to identify significant
plays in real time (on the order of 1-2 minutes), an
improvement over previous approaches that required
access to full dataset. Their identification procedure
did not rely on sentiment classification, but rather
on tweet rates from specific users and grouping 
tweets based on whether they were most likely to have 
been sent by humans or machines (Zhao et al., 2012).\footnote{http://arxiv.org/pdf/1205.3212v1.pdf}. 
This piece of 
previous work provides evidence that tweets respond in
real time to progress in sporting events. We base
our hypothesis that tweets can be predictive and responsive
in the NCAA in real time using this previous result. 

\end{doublespacing}
\end{document}