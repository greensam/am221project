\documentclass[12pt]{article}
\usepackage[hmargin=1in,vmargin=1in]{geometry}
\usepackage{amsmath}
\usepackage{setspace}
\usepackage{graphicx}
\usepackage{sectsty}
\usepackage{fancyvrb}
\usepackage{mathtools}
\usepackage{float}
\usepackage[flushleft]{threeparttable}

% Make all of the environments for different proof elements
\newtheorem{theorem}{Theorem}[section]
\newtheorem{lemma}[theorem]{Lemma}
\newtheorem{proposition}[theorem]{Proposition}
\newtheorem{corollary}[theorem]{Corollary}
\newenvironment{proof}[1][Proof:]{\begin{trivlist}
\item[\hskip \labelsep {\bfseries #1}]}{\end{trivlist}}
\newenvironment{definition}[1][Definition]{\begin{trivlist}
\item[\hskip \labelsep {\bfseries #1}]}{\end{trivlist}}
\newenvironment{example}[1][Example]{\begin{trivlist}
\item[\hskip \labelsep {\bfseries #1}]}{\end{trivlist}}
\usepackage{hyperref}
\usepackage[usenames, dvipsnames]{color}

\definecolor{cuse}{RGB}{212, 69, 0}
\definecolor{uva}{RGB}{0, 55, 119}

% Change back to use Arabic footnotes
\renewcommand*{\thefootnote}{\arabic{footnote}}
\sectionfont{\large}
\subsectionfont{\normalsize}

% Start the document
\begin{document}
\setcounter{page}{1}
\begin{doublespacing}

\title{Draft Conclusion}
\author{David Freed and Samuel Green}
\date{May 1, 2016}
\maketitle

Expanding on previous work applying Twitter for predictive sports 
analysis, this study demonstrates the beneficial link between
relevant Twitter data and both short-term and long-term outcomes 
in NCAA March Madness games. We provided evidence first that 
Twitter is responsive in a short timeperiod to events in games,
both in the volume of tweets sent and in the sentiment contained
in those tweets. We then showed that Twitter data can be used
to make statistically significant predictions about future
game events and showed that, particularly as games progress,
Twitter data can be used to improve standard
models typically used to predict game winners.
These results support previous work demonstrating ``wisdom of the 
crowds'' effects on Twitter and show that Twitter analytics
can be usefully applied to the NCAA. 

Given the significance of our results, we discuss some potential
flaws in the analysis and methodology and avenues for future work.

The foremost point for improvement sits at the pivot point of this
study: sentiment classification. We trained our classifier 
using a corpus of labelled tweets related to Apple and Google
product launches, given the impracticality of hand-constructing
or commissioning a labelled training set for the sports-specific
domain. While hand tests showed that our classifier was 
able to discern conventionally positive language from 
conventionally negative language, this is inherently inadequate
for the sports-specific domain. For example, many words that
are conventionally negative, like ``dirty'' or ``filthy,'' are
often positive in the context of basketball. As an additional
example, our classifier initally called the tweet
\texttt{Hell yes, Syracuse is going to the Final Four}
negative, though it is clearly positive in the context of the 
NCAA tournament. We hypothesis that retraining the model with a 
more appropriate training set would improve model 
performance and highlight this as an obvious opportunity for
future work.

Relevance classification, namely identifying noisey tweets from
tweets that contained information relevant to the game,
remains a challenge. Though we collected tweets only with
specific hashtags, tweets that weren't related to the NCAA
tournament were still collected (which we observed by inspection
of the dataset). We did our best to eliminate these tweets, as 
previously discussed, but finding a mechanism to collect a more
targetted set for classification would likely also contribute
to improvement.

A further training concern is that
our models rely on a noisey projection of game 
events onto real time, since we did not discover a 
source of game data that included real timestamps. 
Our quasi-uniform projection of game time onto real time
results in a reasonable approximation, but he models
should be reconstructed 
using more reliable game data if this becomes possible
in the future. 

This setting provides rich opportunity for extensions
and future work. We first note that it would be interesting
to consider a more granual seperation of the ``crowd'' into
trusted and untrusted ``sub-crowds.'' That is, could
predictive improvements be made by considering crowd-sourced
opinions from known experts, like commentators or individuals
prominent in the network, separate from or more heavily weighted
than standard or low influence individuals? Presumably, experts
are ex ante more qualified to predict outcomes based on 
game progress. Second, our dataset was not large enough
or practically enough to labelled to allow for analysis
of Twitter's relationship to non-scoring events: for
example, links between sentiment and probabilities of 
a foul or an injury occuring, each of which might seem ex ante
correlated to the overall momentum that we have shown Twitter
is able to approximate. Re-application to an NBA dataset
would also be natural. 

To our knowledge, the results presented here are novel.
We provide the first models using Twitter data as input to a real
-time model. We are also the first to show that a model of this
variety has power when predicting future events and full game
outcomes in the NCAA. 

\end{doublespacing}
\end{document}





























